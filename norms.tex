
\section{Matrix Norms}
\label{sec:matrix-norms}

Before I can discuss the norm of a matrix, I should first describe the norm of a vector. A
vector's \newterm{norm} is a measure of the vector's length. For example, if the components of a
vector $X$ are $x_1$, $x_2$, and $x_3$ the usual formula for the vector's length is $L(X) =
\sqrt{x_1^2 + x_2^2 + x_3^2}$. This formula has the following ``length-like'' properties.

\begin{enumerate}

\item $L(X) \ge 0$, with $L(X) = 0$ if and only if $X = 0$.

\item $L(X + Y) \le L(X) + L(Y)$.

\item $L(\alpha X) = \abs{\alpha}\, L(X)$.

\end{enumerate}

The second property above is called the triangle inequality. It says that the length of the sum
of two vectors is less than or equal to the sum of the two lengths. A simple diagram will make
it obvious that this is true in the case of the traditional concept of length.

The vector function $L(X)$ above satisfies the length-like properties but it is not the only
function that does so. Any function that satisfies those properties (for all vectors) can be
considered a ``length''. Such functions are norms. In fact, $L(X)$ above is actually one
function in an entire class of functions called \newterm{p-norms}. The general form of a p-norm
is
\begin{displaymath}
\norm{X}_p = (\abs{x_1}^p + \abs{x_2}^p + \cdots + \abs{x_n}^p)^{\frac{1}{p}},\; p \ge 1
\end{displaymath}

We use the symbol $\norm{X}_p$ to represent the p-norm of vector $X$. You can see that the
traditional formula for the length of a vector is the 2-norm of the vector. In addition to the
2-norm we will also be interested in the 1-norm and the $\infty$-norm. The $\infty$-norm is the
result of a limiting process and is found from
\begin{displaymath}
\norm{X}_\infty = \max_{1 \le i \le n} \abs{x_i}
\end{displaymath}
That is the $\infty$-norm is simply the component of the vector with the largest absolute value.

We can now extend these concepts to matrices. One way to do that would be to treat a $m \times
n$ matrix as a vector with $mn$ components. One could then simply use a suitable vector norm as
a matrix norm. It is easy to see that a matrix norm defined in this way will continue to have
all the required properties provided that the underlying vector norm has those properties.

However, since the product of two $n \times n$ matrices is again an $n \times n$ matrix, it
turns out to be useful to add an additional constraint on the behavior of matrix norms. In
particular, we will want $\norm{AB} \le \norm{A} \cdot \norm{B}$. With this condition in place
the simple extension of vector norms to matrix norms does not work.

However, one can define useful matrix norms in terms of vector norms anyway. In particular, let
the norm of a matrix $A$ be
\begin{displaymath}
\norm{A} = \max_{\norm{u} = 1} \norm{Au}
\end{displaymath}

To understand what this means remember that a matrix, when multiplied by a vector, performs a
linear transformation of that vector into another vector. In the above expression the vector $u$
has a norm of 1 and is thus a \newterm{unit vector}. There are, in general, infinitely many unit
vectors (depending on the norm used). For example, if one uses the 2-norm the unit vectors
describe a sphere of radius 1 centered on the origin. The norm of a matrix is thus the norm of
the ``largest'' vector produced by applying the matrix as a linear transformation to every
possible unit vector. For example, if one used a 2-norm and applied the matrix to every point on
the surface of a unit sphere, the norm of the matrix would be the distance from the origin to
the most distant point on the resulting surface.

It is important to note that any vector norm can be used to define a matrix norm in this way.
Then when one speaks, for example, of the p-norm of a matrix one is talking about the matrix
norm that results from using the corresponding p-norm of a vector. This method of defining a
matrix norm may seem odd but it has a number of useful properties. For an $m \times n$ matrix
\begin{eqnarray*}
\norm{A}_1 & = & \max_{1 \le j \le n} \sum_{i = 1}^m \abs{a_{ij}} \\
\norm{A}_\infty & = & \max_{1 \le j \le m} \sum_{j = 1}^n \abs{a_{ij}}
\end{eqnarray*}

In other words the 1-norm is the maximum of the sums of the columns, with the understanding that
it is the sums of the absolute values of the elements that are done. Similarly the $\infty$-norm
is the maximum of the sums of the rows, with the same understanding about taking the absolute
values of the elements. For example, the matrix below shows the column sums below each column
and the row sums beside each row.
\begin{eqnarray*}   % I don't think I know what I'm doing here. There must be a better way.
&
\left[
\begin{array}{rrr}
 1.23 & -3.42 &  0.89 \\
-4.48 &  4.56 &  6.22 \\
 5.19 &  3.95 & -2.88
\end{array}
    \right] &
\left.
\begin{array}{rr}
\rightarrow &  5.54 \\
\rightarrow & 15.26 \\
\rightarrow & 12.02
\end{array}
\right.\\
&
\begin{array}{rrr}
\downarrow & \downarrow & \downarrow \\
10.90 & 11.93 & 9.99
\end{array} &
\end{eqnarray*}

Thus the 1-norm of the matrix above is 11.93 and the $\infty$-norm of the matrix is 15.26.
Notice that if a single element in the matrix was much larger than any of the others, it would
dominate its column sum and its row sum. That element would become, in effect, both the 1-norm
and the $\infty$-norm.

It is important to realize that both the 1-norm and the $\infty$-norm of a matrix can be
calculated in $\Theta(n^2)$ time. As you will see in Section~\ref{sec:simul-equations}, this is
more efficient than what is required to invert a general matrix. This observation is important
during the solution of simultaneous equations.

\subsection*{Exercises}

\begin{enumerate}

\item \label{exer:compute-vnorm} Calculate the 1-norm, 2-norm, 3-norm, and $\infty$-norm of the
  following vectors

\begin{enumerate}
  \item $(1.27, -3.12, 4.84)$
  \item $(3.44, -2.19, 15.32)$
  \item $(0.00, 0.00, 0.00)$
\end{enumerate}

\item Prove that for any vector $X$, $\norm{X}_\infty \le \norm{X}_p$ for all finite $p$. Prove
  that if $q > p$, then $\norm{X}_q \le \norm{X}_p$.

\item \label{exer:compute-mnorm} Calculate the 1-norm and $\infty$-norm of the following matrix.
\begin{displaymath}
\left[
\begin{array}{rrrr}
6.28 & -12.31 &   4.22 &  1.33 \\
4.89 &  -2.23 &   6.69 & -3.44 \\
8.32 &   2.41 & -10.34 &  9.83 \\
0.48 &  -0.48 & -18.40 &  4.44
\end{array}
\right]
\end{displaymath}

\item \label{exer:transpose-norm} How do the p-norms of the transpose of a matrix relate to the
  p-norms of the original matrix?

\end{enumerate}

\subsection*{Octave}

Octave has a built in \texttt{norm} function that can be used to compute the norm of a vector or
of a matrix. Read about it in Octave's online help. Use that function to check your answers in
Exercise~\ref{exer:compute-vnorm} and Exercise~\ref{exer:compute-mnorm} above. Use Octave to
compute the norms of several matrices and their transposes and see if you can verify your
findings in Exercise~\ref{exer:transpose-norm}.
