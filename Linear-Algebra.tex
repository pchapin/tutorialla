%%%%%%%%%%%%%%%%%%%%%%%%%%%%%%%%%%%%%%%%%%%%%%%%%%%%%%%%%%%%%%%%%%%%%%%%%%%%
% FILE         : Linear-Algebra.tex
% AUTHOR       : (C) Copyright 2010 by Peter C. Chapin
% SUBJECT      : Quick primer on the basics of linear algebra.
%
% To Do:
%
% + See the comments in the text for ideas about places where this document could be improved.
%%%%%%%%%%%%%%%%%%%%%%%%%%%%%%%%%%%%%%%%%%%%%%%%%%%%%%%%%%%%%%%%%%%%%%%%%%%%

%+++++++++++++++++++++++++++++++++
% Preamble and global declarations
%+++++++++++++++++++++++++++++++++
\documentclass{article}
\setlength{\parindent}{0em}
\setlength{\parskip}{1.75ex plus0.5ex minus0.5ex}

\usepackage{hyperref}

% Useful commands to streamline my typing.
\newcommand{\abs}[1]{|#1|}
\newcommand{\norm}[1]{\|#1\|}

\newtheorem{theorem}{Theorem}[section]

%++++++++++++++++++++
% The document itself
%++++++++++++++++++++
\begin{document}

% Nice command for formatting new terms.
\newcommand{\newterm}[1]{\emph{#1}}

%-----------------------
% Title page information
%-----------------------
\title{Linear Algebra}
\author{Peter C. Chapin}
\date{February 6, 2010}
\maketitle

\tableofcontents

\section*{Legal}
\label{sec:legal}

\textit{Permission is granted to copy, distribute and/or modify this document under the terms of
  the GNU Free Documentation License, Version 1.1 or any later version published by the Free
  Software Foundation; with no Invariant Sections, with no Front-Cover Texts, and with no
  Back-Cover Texts. A copy of the license is included in the file \texttt{GFDL.txt} distributed
  with the \LaTeX\ source of this document.}

\section{Introduction}
\label{sec:intro}

This document is a quick primer on linear algebra. Its intended audience is senior computer
engineering technology students at Vermont Technical College. This document assumes the reader
has a reasonable mathematical background, largely calculus based, but no prior exposure to the
topics of linear algebra at all.

At the end of each major section below you will find some exercies that will encourage you to
practice with the concepts presented in that section. In addition you will find some notes on
how to use Octave to experiment with the topics of the section. Octave is an open source program
that excels at doing matrix computations of various kinds (as well as many other mathematical
things). It is similar to the commerical program MATLAB and even uses a largely compatible
syntax. You may also be able to do a number of interesting matrix computations on your
calculator. However, Octave will probably have greater capacity, higher performance, and more
functionality than your calculator. Regardless of what your calculator can or can't do, I
recommend that you spend some time getting familiar with Octave.


\section{The Concept of a Matrix}
\label{sec:matrix-concept}

A \newterm{matrix} is, essentially, a rectangular array of numbers. It is the basic ``data
type'' used in linear algebra. The individual numbers in a matrix, usually called the
\newterm{elements}, can be drawn from any of the usual sets. In many applications the matrix
elements are real numbers but in some cases they might be complex numbers, integers, or rational
numbers. We speak of real matrices, complex matrices, and so forth according to the type of the
matrix elements.

The dimensions of a matrix are usually given in the form $m \times n$ where $m$ is the number of
rows and $n$ is the number of columns. The row dimension is always given first. If $n = m$ then
the matrix is said, for obvious reasons, to be square. Here is an example of how a $2 \times 3$
matrix named $A$ might be displayed.

\begin{displaymath}
A = \left[
\begin{array}{ccc}
a_{11} & a_{12} & a_{13} \\
a_{21} & a_{22} & a_{23}
\end{array}
    \right]
\end{displaymath}

Each element is given two subscripts to indicate its position in the matrix. The first subscript
is the row number, ranging from $1\ldots m$, and the second subscript is the column number,
ranging from $1\ldots n$. Typically when one wants to refer to a generic element of matrix one
would use variables such as $i$, $j$, or $k$ to be placeholders for the indices. For example, a
typical element of matrix $A$ might be written as $a_{ij}$. As a computer programmer you should
be able to relate this notation to the index variables used in typical loops. For example, in C
\begin{verbatim}
double A[2][3];

for (int i = 0; i < 2; i++) {
  for (int j = 0; j < 3; j++) {
    A[i][j] = ... ;
  }
}
\end{verbatim}

Notice that in the mathematical notation it is traditional to regard the matrix subscripts as
starting at one and not zero. Also in the mathematical notation it is traditional to use
uppercase letters to represent the entire matrix and lowercase letters to represent individual
matrix elements.

Matrices with only one row, that is $1 \times n$ matrices, are often called \newterm{row
  vectors}. In many cases the row subscript is dropped, since it is always $1$, when talking
about the elements of a row vector. Similarly matrices with only one column are often called
\newterm{column vectors}. In that case the column subscript is often dropped when talking about
the elements of a column vector. Row vectors and column vectors are essentially ordinary arrays
in a programming sense. In the mathematics of linear algebra they are basically treated like any
other matrix although their unusual dimensions give them somewhat special properties.

You might have noticed that in the notation above I did not put any punctuation between the two
subscripts in, for example, $a_{12}$. This might cause you to wonder what happens if one or both
of the subscripts exceeds $9$. In general this is not a significant concern. In the mathematical
development one rarely uses specific numbers other than $1$ or $2$ anyway. For example, I might
show a general $m \times n$ matrix as follows

\begin{displaymath}
A = \left[
\begin{array}{cccc}
  a_{11} & a_{12} & \cdots & a_{1n} \\
  a_{21} & a_{22} & \cdots & a_{2n} \\
  \vdots & \vdots & \ddots & \vdots \\
  a_{m1} & a_{m2} & \cdots & a_{mn}
\end{array}
    \right]
\end{displaymath}

In the discussion about this matrix I might talk about element $a_{ij}$ and how, for example, it
relates to $a_{ii}$ or $a_{mj}$ or even $a_{i+1,j-1}$. However, it would be extremely rare for
me to talk about a specific element such as $a_{3982}$ so the ambiguity about what $3982$ means
in this case would not be an issue.

Keep in mind also that many of the applications of linear algebra involve the use of very large
matrices. While the examples in this document and most text books are small so that they can be
worked by hand, in the real world matrices with dimensions of many thousands of rows and columns
are commonplace. The theory, of course, has no special problems with such large matrices but
doing computations efficiently on them can be a serious concern. Much has been written about
that subject but it is outside the scope of this document.

\subsection*{Exercises}

\begin{enumerate}

\item \emph{I need something here.}

\end{enumerate}

\subsection*{Octave}

I will assume here that you will be using the open source program Octave for your computer
experiments. However, if you have access to MATLAB you should find that all of the commands I
talk about here will also work for you. The two programs have a substantial degree of
compatibility, particularly in the area of basic commands.

When you start Octave you will get a prompt that might look something like

\begin{verbatim}
octave:1>
\end{verbatim}

The number (shown as ``1'' above) is the command number. It increments after each command
although in this document I will typically show it always as ``1.'' You can use octave as a
simple calculator, just type in expressions for it to evaluate. For example

\begin{verbatim}
octave:1> 2 + 10 / 5
ans = 4
octave:2>
\end{verbatim}

Octave stores the result of the evaluation into a special variable named \texttt{ans} that it
displays on the next line. After displaying the answer it then prints a new prompt. You can
store the result of an expression into a named variable using the obvious syntax.

\begin{verbatim}
octave:1> x = 2 + 10 / 5
x = 4
\end{verbatim}

The beauty of Octave is that it allows you to manupulate matrices in a natural way. To enter a
matrix use square brackets.

\begin{verbatim}
octave:1> A = [ 1 2 3; 4 5 6; 7 8 9 ]
A =

  1   2   3
  4   5   6
  7   8   9
\end{verbatim}

Octave allows a fairly flexible input syntax. You can separate matrix elements with white space
alone or with commas at your option. The semicolons are used to delimit one row from another.
However, you could have also entered the matrix on multiple lines using the RETURN key at the
end of each row in a natural way.

Once a matrix has been entered you can access an individual element using parenthesis. For
example \texttt{A(1,3)} prints out the value of $a_{13}$. You can also modify a particular
matrix element.

\begin{verbatim}
octave:2> A(1,3) = 10
A =

  1   2  10
  4   5   6
  7   8   9
\end{verbatim}

Notice that when a matrix element is modified, Octave redisplays the entire matrix.

There are some quirks with Octave to keep in mind as you use it. First, as is mathematically
traditional, element indices are numbered starting at $1$, not $0$ as a C programmer might
expect. Also Octave uses parenthesis instead of square brackets when accessing an individual
element. These things are consistent with Fortran's syntax and are considered natural among
numerical specialists who use Fortran. Also be aware that Octave uses a floating point type
internally to hold all scalars. This is typically what you want but it means that its integer
calculations are somewhat of an illusion. Be careful.

You should take Octave for a test drive. Try entering in a few matrices and accessing their
elements. Although I didn't discuss it here, experiment with Octave's slice notation. For
example, using the matrix $A$ entered above, try executing the command \texttt{A(2:3, 2)}. Can
you explain what happens? Verify your explaination by trying some other possibilities. Also try
\texttt{A(:, 2)} as well.


\section{Matrix Operations}
\label{sec:matrix-operations}

Various operations have been defined for matrices so that they may be manipulated as a single
unit. In this section I will review some of those basic operations. Please read this section
carefully. Understanding this material is essential for understanding the later material.

Two matrices can be added if they have the same dimensions. If the dimensions are different the
addition of the matrices is undefined. To add two compatible matrices one simply adds
corresponding elements. Thus addition is very natural. For example

\begin{displaymath}
\left[
\begin{array}{cc}
  a_{11} & a_{12} \\
  a_{21} & a_{22}
\end{array}
\right] +
\left[
\begin{array}{cc}
  b_{11} & b_{12} \\
  b_{21} & b_{22}
\end{array}
\right] =
\left[
\begin{array}{cc}
  a_{11} + b_{11} & a_{12} + b_{12} \\
  a_{21} + b_{21} & a_{22} + b_{22}
\end{array}
\right]
\end{displaymath}

To multiply a matrix by a scalar, simply multiple each element in the matrix by that scalar.
This is also very natural. For example

\begin{displaymath}
a \left[
\begin{array}{cc}
  b_{11} & b_{12} \\
  b_{21} & b_{22}
\end{array}
\right] =
\left[
\begin{array}{cc}
  a b_{11} & a b_{12} \\
  a b_{21} & a b_{22}
\end{array}
\right]
\end{displaymath}

Matrix subtraction follows from the above two operations. If $A$ and $B$ are matrices then $A -
B$ can be regarded as $A + (-1)B$. Thus matrix subtraction is also done on an element by element
basis in a natural way.

Because matrix addition is defined as element by element addition it should be fairly clear that
matrix addition obeys the same laws of commutivity and associativity as ordinary addition. That
is, if $A$, $B$, and $C$ are matrices then $A + B = B + A$ and $(A + B) + C = A + (B + C)$. This
assumes that all the matrices involved have the same dimensions so that the addition operations
are properly defined.

Matrix multiplication, on the other hand, is defined in a way that might seem rather
counter-intuitive at first. However, as you will see later, it actually provides a very useful
operation. Let the matrices to be multipled be called $A$ and $B$. Let $C = AB$ be the product
matrix. Let $A$ have the dimensions $m \times n$ and $B$ have the dimensions $n \times p$. For
the multiplication to be defined the two inner dimensions must agree. The dimension of $C$ is
then given by the two outer dimensions. In this case that would be $m \times p$. For example you
could multiple a $3 \times 2$ matrix with a $2 \times 5$ matrix. The result would be a $3 \times
5$ matrix.

To compute an element in the result matrix, say $c_{ij}$, you would run down row $i$ of $A$ and
column $j$ of $B$. Because of the rules described above, there would be $n$ columns in $A$ and
$n$ rows in $B$ so the number of items in each run would be the same. You calculate $c_{ij}$ as
follows

\begin{displaymath}
c_{ij} = \sum_{k = 1}^{n} a_{ik} b_{kj}
\end{displaymath}

The computation is, essentially the vector dot product of the $i^{\mathrm{th}}$ row vector of
matrix $A$ and the $j^{\mathrm{th}}$ column vector of matrix $B$. Of course you would have to
repeat this calculation, using different rows and columns in $A$ and $B$ respectively, for each
element in the result matrix.

As an example consider the following product of a $3 \times 2$ matrix with a $2 \times 3$
matrix.

\begin{displaymath}
\left[
\begin{array}{rr}
 1  &  2 \\
 4  & -1 \\
-3  &  3
\end{array}
\right] \times
\left[
\begin{array}{rrr}
-2 & 3 & 1 \\
 5 & 2 & 5
\end{array}
\right] =
\left[
\begin{array}{rrr}
   8 &  7 & 11 \\
 -13 & 10 & -1 \\
  21 & -3 & 12 
\end{array}
\right]
\end{displaymath}

For example $c_{11} = a_{11}b_{11} + a_{12}b_{21} = (1)(-2) + (2)(5) = 8$. The other elements of
the result matrix are calculated similarly. You should verify them to be sure you understand how
matrix multiplication works. Notice also that in this case the result matrix is larger than
either of the matrices being multiplied together. This is a consequence of dimensioning rules.
It won't always be like that, however. For example a $2 \times 6$ matrix being multiplied with a
$6 \times 2$ matrix will result in a smaller $2 \times 2$ matrix. In the special case where all
the matrices involved are square the dimensions will stay the same during the multiplication.

Matrix multiplication is not commutative. For one thing reversing the order of the matrices
might cause the multiplication to be undefined due to incompatible dimensions. Even if the
multiplication is defined the result matrix might not be the same size. For example, reversing
the order of the matrices in the last example will produce a $2 \times 2$ result, not a $3
\times 3$ result. Even if the result matrices are the same size (both $A$ and $B$ are square),
the results will not, in general, be the same. It is easy to find two square matrices that don't
commute when multiplied. Most pairs don't.

Matrix multiplication does, however, associate. That is if $A$, $B$, and $C$ are matrices $(AB)C
= A(BC)$. As a consequence of this the parenthesis are not necessary when writing the product of
several matrices. It also means that something like $A^3$ is well defined. In particular, $A^3 =
AAA = (AA)A = A(AA)$. Notice that only square matrices can be raised to powers (do you see
why?).

At this point you might be wondering about the definition of matrix division. It turns out that
the problem of dividing matrices is fraught with subtle complications. In fact we do not define
it at all in the usual sense. Instead we talk about matrix inverses. However, that is a topic
that deserves its own section.

\subsection*{Exercises}

\begin{enumerate}

\item Demonstrate that matrix multiplication, even of square matrices, is not in general
  commutative.

\item Prove that matrix multiplication is associative.

\item Does matrix multiplication distribute over matrix addition? That is, if $A$, $B$, and $C$
  are matrices does $A(B + C) = AB + AC$? If so, then prove it. If not, then find a counter
  example using Octave (see below). Repeat the question for $(B + C)A = BA + CA$.

\end{enumerate}

\subsection*{Octave}

Once matrices are defined in Octave you can add, subtract, and multiply them with the obvious
expressions. Octave will print the results of each expression as you enter it (you can supress
such printing by ending your command with a semicolon). Use Octave to demonstrate the
commutativity and associativity (or lack thereof) of matrix addition and matrix multiplication.
Use Octave to support your answer to the distribution question above. Either find suitable
counter examples or demonstrate that distribution does work.

Keep in mind that numerical experiments of the sort you can do in Octave can prove that a
certain property does not hold in all cases by finding a single counter example. You do not need
to try every matrix in existence. However, Octave can not prove that a certain property does
hold in all cases since it can't test every possible case. One must rely on the standard
techniques of mathematical proof to verify that a property is true in all cases.

More specifically, you can use Octave to prove that matrix multiplication does not commute by
finding two matrices that don't commute. However, you can't prove that matrix multiplication is
associative without computing $A(BC)$ and $(AB)C$ for every possible $A$, $B$, and $C$. That
would require an infinite amount of computation since there are infinitely many matrices.
However, you can \emph{demonstrate} the association of matrix multiplication in Octave by
computing $A(BC)$ and $(AB)C$ for some particular $A$, $B$, and $C$ and then showing that the
results are the same. Such a demonstration does not constitute a proof of the general case but
it is interesting anyway.


\section{Linear Transformations}
\label{sec:linear-transform}

\subsection*{Exercises}

\begin{enumerate}

\item \emph{I need something here.}

\end{enumerate}

\subsection*{Octave}


\section{Matrix Inverses}
\label{sec:matrix-inverses}

In order to talk about the inverse of a matrix I should first talk about the concept of
identities. Consider ordinary multiplication. There exists a number, namely $1$, that has the
property $x \times 1 = 1 \times x = x$. Thus $1$ is called the \newterm{multiplicative identity}
because when you multiply any number by $1$ the result is the same. It turns out that there is a
\newterm{identity matrix} as well. Actually there are infinitely many identity matrices of
different sizes. The general form is
\begin{displaymath}
I_n = \left[
\begin{array}{cccc}
 1 & 0 & \cdots & 0 \\
 0 & 1 & \cdots & 0 \\
 \vdots & \vdots & \ddots & \vdots \\
 0 & 0 & \cdots & 1
\end{array}
\right]
\end{displaymath}

That is, $I_n$ is an $n \times n$ square matrix that has ones down the diagonal and zeros
everywhere else. It is relatively easy to see that $IA = A$ for any matrix $A$ (just try
multiple out an example and you will quickly see how it works). Note that $A$ need not be
square. As long as you use the proper sized identity matrix so that the multiplication is
defined the result of $IA$ will be $A$ again.

The identity matrix is also special because, unlike the general case, multiplication by the
identity martix commutes. That is $IA = AI = A$. This assumes that $A$ is square. Otherwise one
must use different sized identities depending on if the identity is being pre-multiplied or
post-multiplied.

Another property of real numbers is that every number except zero has a \newterm{multiplicative
  inverse}. For example $x$'s inverse, symbolized as $x^{-1}$ is such that $xx^{-1} = x^{-1}x =
1$. Similarly many matrices have inverses as well. For example $A$'s inverse, symobolized as
$A^{-1}$ is such that $AA^{-1} = A^{-1}A = I$. Here I restrict myself to square matrices. In
that case notice how multiplication by the inverse commutes; again contrary to the general case
for matrix multiplication.

% It might be useful to prove some of these properties.

With real numbers we can define division in terms of multiplication by an inverse. Specifically
we can define $a/b$ as $ab^{-1}$ or as $b^{-1}a$. Here I'm taking advantage of the fact that
multiplication of real numbers commutes in every case. Similarly in situations where one is
tempted to divide by a matrix we normally talk instead about multiplying by the inverse of the
matrix instead. Because matrix multiplication is not commutative we have the added complication,
in the algebra of matrices, of having to carefully track pre-multiplications as opposed to
post-multiplications. An example of this will be shown in the next section.

Computing the inverse of a matrix is a non-trivial problem. I will discuss this more in a later
version of this document.

% Not all matrices have inverses. Some matrices are singular. Define and discuss this. It might
% also be necessary to talk about the det() operation and how it relates to the existence of an
% inverse.

\subsection*{Exercises}

\begin{enumerate}

\item Consider the following $n \times n$ matrix.
\begin{displaymath}
A = \left[
\begin{array}{cccc}
  a_{11} & 0 & \cdots & 0 \\
  0 & a_{22} & \cdots & 0 \\
  \vdots & \vdots & \ddots & \vdots \\
  0 & 0 & \cdots & a_{nn}
\end{array}
    \right]
\end{displaymath}
This matrix has non-zero elements only on its main diagonal. What is its inverse?

\end{enumerate}

\subsection*{Octave}

You can use the \texttt{eye()} function in Octave to generate an identity matrix of any size.
For example \texttt{eye(3)} returns a $3 \times 3$ identity. You can also use the \texttt{inv()}
function to compute the inverse of a matrix. For example

\begin{verbatim}
octave:1> A = [ 1 2 4; -4 2 1; 9 -2 1 ];
octave:2> B = inv(A)
\end{verbatim}

Verify that the result of \texttt{inv()} is a proper inverse by computing $A*B$ and $B*A$ and
showing that both result in the identity matrix. You may notice that the results shows slight
round off errors in the computations (mostly identified by the existence of negative zeros in
the result). This is normal.

What does Octave do when asked to find the inverse of the singular matrix in the example above?


\section{Matrix Norms}
\label{sec:matrix-norms}

Before I can discuss the norm of a matrix, I should first describe the norm of a vector. A
vector's \newterm{norm} is a measure of the vector's length. For example, if the components of a
vector $X$ are $x_1$, $x_2$, and $x_3$ the usual formula for the vector's length is $L(X) =
\sqrt{x_1^2 + x_2^2 + x_3^2}$. This formula has the following ``length-like'' properties.

\begin{enumerate}

\item $L(X) \ge 0$, with $L(X) = 0$ if and only if $X = 0$.

\item $L(X + Y) \le L(X) + L(Y)$.

\item $L(\alpha X) = \abs{\alpha}\, L(X)$.

\end{enumerate}

The second property above is called the triangle inequality. It says that the length of the sum
of two vectors is less than or equal to the sum of the two lengths. A simple diagram will make
it obvious that this is true in the case of the traditional concept of length.

The vector function $L(X)$ above satisfies the length-like properties but it is not the only
function that does so. Any function that satisfies those properties (for all vectors) can be
considered a ``length''. Such functions are norms. In fact, $L(X)$ above is actually one
function in an entire class of functions called \newterm{p-norms}. The general form of a p-norm
is
\begin{displaymath}
\norm{X}_p = (\abs{x_1}^p + \abs{x_2}^p + \cdots + \abs{x_n}^p)^{\frac{1}{p}},\; p \ge 1
\end{displaymath}

We use the symbol $\norm{X}_p$ to represent the p-norm of vector $X$. You can see that the
traditional formula for the length of a vector is the 2-norm of the vector. In addition to the
2-norm we will also be interested in the 1-norm and the $\infty$-norm. The $\infty$-norm is the
result of a limiting process and is found from
\begin{displaymath}
\norm{X}_\infty = \max_{1 \le i \le n} \abs{x_i}
\end{displaymath}
That is the $\infty$-norm is simply the component of the vector with the largest absolute value.

We can now extend these concepts to matrices. One way to do that would be to treat a $m \times
n$ matrix as a vector with $mn$ components. One could then simply use a suitable vector norm as
a matrix norm. It is easy to see that a matrix norm defined in this way will continue to have
all the required properties provided that the underlying vector norm has those properties.

However, since the product of two $n \times n$ matrices is again an $n \times n$ matrix, it
turns out to be useful to add an additional constraint on the behavior of matrix norms. In
particular, we will want $\norm{AB} \le \norm{A} \cdot \norm{B}$. With this condition in place
the simple extension of vector norms to matrix norms does not work.

However, one can define useful matrix norms in terms of vector norms anyway. In particular, let
the norm of a matrix $A$ be
\begin{displaymath}
\norm{A} = \max_{\norm{u} = 1} \norm{Au}
\end{displaymath}

To understand what this means remember that a matrix, when multiplied by a vector, performs a
linear transformation of that vector into another vector. In the above expression the vector $u$
has a norm of 1 and is thus a \newterm{unit vector}. There are, in general, infinitely many unit
vectors (depending on the norm used). For example, if one uses the 2-norm the unit vectors
describe a sphere of radius 1 centered on the origin. The norm of a matrix is thus the norm of
the ``largest'' vector produced by applying the matrix as a linear transformation to every
possible unit vector. For example, if one used a 2-norm and applied the matrix to every point on
the surface of a unit sphere, the norm of the matrix would be the distance from the origin to
the most distant point on the resulting surface.

It is important to note that any vector norm can be used to define a matrix norm in this way.
Then when one speaks, for example, of the p-norm of a matrix one is talking about the matrix
norm that results from using the corresponding p-norm of a vector. This method of defining a
matrix norm may seem odd but it has a number of useful properties. For an $m \times n$ matrix
\begin{eqnarray*}
\norm{A}_1 & = & \max_{1 \le j \le n} \sum_{i = 1}^m \abs{a_{ij}} \\
\norm{A}_\infty & = & \max_{1 \le j \le m} \sum_{j = 1}^n \abs{a_{ij}}
\end{eqnarray*}

In other words the 1-norm is the maximum of the sums of the columns, with the understanding that
it is the sums of the absolute values of the elements that are done. Similarly the $\infty$-norm
is the maximum of the sums of the rows, with the same understanding about taking the absolute
values of the elements. For example, the matrix below shows the column sums below each column
and the row sums beside each row.
\begin{eqnarray*}   % I don't think I know what I'm doing here. There must be a better way.
&
\left[
\begin{array}{rrr}
 1.23 & -3.42 &  0.89 \\
-4.48 &  4.56 &  6.22 \\
 5.19 &  3.95 & -2.88
\end{array}
    \right] &
\left.
\begin{array}{rr}
\rightarrow &  5.54 \\
\rightarrow & 15.26 \\
\rightarrow & 12.02
\end{array}
\right.\\
&
\begin{array}{rrr}
\downarrow & \downarrow & \downarrow \\
10.90 & 11.93 & 9.99
\end{array} &
\end{eqnarray*}

Thus the 1-norm of the matrix above is 11.93 and the $\infty$-norm of the matrix is 15.26.
Notice that if a single element in the matrix was much larger than any of the others, it would
dominate its column sum and its row sum. That element would become, in effect, both the 1-norm
and the $\infty$-norm.

It is important to realize that both the 1-norm and the $\infty$-norm of a matrix can be
calculated in $\Theta(n^2)$ time. As you will see in Section~\ref{sec:simul-equations}, this is
more efficient than what is required to invert a general matrix. This observation is important
during the solution of simultaneous equations.

\subsection*{Exercises}

\begin{enumerate}

\item \label{exer:compute-vnorm} Calculate the 1-norm, 2-norm, 3-norm, and $\infty$-norm of the
  following vectors

\begin{enumerate}
  \item $(1.27, -3.12, 4.84)$
  \item $(3.44, -2.19, 15.32)$
  \item $(0.00, 0.00, 0.00)$
\end{enumerate}

\item Prove that for any vector $X$, $\norm{X}_\infty \le \norm{X}_p$ for all finite $p$. Prove
  that if $q > p$, then $\norm{X}_q \le \norm{X}_p$.

\item \label{exer:compute-mnorm} Calculate the 1-norm and $\infty$-norm of the following matrix.
\begin{displaymath}
\left[
\begin{array}{rrrr}
6.28 & -12.31 &   4.22 &  1.33 \\
4.89 &  -2.23 &   6.69 & -3.44 \\
8.32 &   2.41 & -10.34 &  9.83 \\
0.48 &  -0.48 & -18.40 &  4.44
\end{array}
\right]
\end{displaymath}

\item \label{exer:transpose-norm} How do the p-norms of the transpose of a matrix relate to the
  p-norms of the original matrix?

\end{enumerate}

\subsection*{Octave}

Octave has a built in \texttt{norm} function that can be used to compute the norm of a vector or
of a matrix. Read about it in Octave's online help. Use that function to check your answers in
Exercise~\ref{exer:compute-vnorm} and Exercise~\ref{exer:compute-mnorm} above. Use Octave to
compute the norms of several matrices and their transposes and see if you can verify your
findings in Exercise~\ref{exer:transpose-norm}.


\section{Simultaneous Equations}
\label{sec:simul-equations}

One of the most important applications of matrices and linear algebra is in the study of systems
of simultaneous linear equations.

A system of two linear equations and two unknowns can always be written in the following general
way.
\begin{eqnarray*}
	a_{11}x_1 + a_{12}x_2 & = & b_1 \\
	a_{21}x_1 + a_{22}x_2 & = & b_2
\end{eqnarray*}

The unknowns are $x_1$ and $x_2$. In many elementary texts the unknowns are often referred to as
$x$ and $y$ but that notation is less extensible to the more general case of many unknowns. The
equations above are called \newterm{linear equations} because each equation taken alone defines
a straight line on an $x_1$ vs $x_2$ plane. The intersection of those two lines defines a single
point, $(x_1, x_2)$, that satisfies both equations simultaneously.

One is usually interested in solving the system of equations for the two unknowns given values
for the coefficents and for $b_1$ and $b_2$. However, before worrying about how to best solve
such systems, it is useful to make the following observation. Let
\begin{displaymath}
x = \left[
\begin{array}{c}
  x_1 \\
  x_2
\end{array}
\right]
\end{displaymath}
Call $x$ the \newterm{vector of unknowns}. Notice that it is a $2 \times 1$ column vector. Let
\begin{displaymath}
b = \left[
\begin{array}{c}
  b_1 \\
  b_2
\end{array}
\right]
\end{displaymath}
Call $b$ the \newterm{driving vector}. It is also a $2 \times 1$ column vector consisting of the
constant terms in the system of equations. Finally let
\begin{displaymath}
A = \left[
\begin{array}{cc}
a_{11} & a_{12} \\
a_{21} & a_{22}
\end{array}
\right]
\end{displaymath}
Call $A$ the \newterm{matrix of coefficients}. It is a $2 \times 2$ square matrix.

Now the system of equations can be succintly expressed by the matrix equation
\begin{displaymath}
Ax = b
\end{displaymath}
The expression $Ax$ is the multiplication of a $2 \times 2$ matrix with a $2 \times 1$ column
vector. By the rules of matrix multiplication this is properly defined. The result will be
another $2 \times 1$ column vector---exactly the dimensions of $b$. The first element in $b$, in
row $1$, column $1$, can be found by scanning down row $1$ of $A$ and column $1$ of $x$ and
forming the sum $a_{11}x_1 + a_{12}x_2$. Thus
\begin{displaymath}
b_1 = a_{11}x_1 + a_{12}x_2
\end{displaymath}

This is, of course, the first equation in the system of equations. The other equation follows
similarly. Now we can see how the odd definition of matrix multiplication can be useful!

A linear system of three equations and three unknowns can be written
\begin{eqnarray*}
a_{11}x_1 + a_{12}x_2 + a_{13}x_3 = b_1 \\
a_{21}x_1 + a_{22}x_2 + a_{23}x_3 = b_2 \\
a_{31}x_1 + a_{32}x_2 + a_{33}x_3 = b_3
\end{eqnarray*}

Each equation in such a system defines a plane in three dimensional space with coordinates
$x_1$, $x_2$, and $x_3$. Two planes intersect to define a line and the intersection of that line
with the third plane defines a single point. That point is the solution of the system; it is the
specific value of $(x_1, x_2, x_3)$ that satisfies all three equations simultaneously. Although
this system is larger, it can still be expressed with the matrix equation $Ax = b$ with the
understanding that $A$ is a $3 \times 3$ matrix in this case.

What of very large matrices? Suppose $A$ was a general $n \times n$ matrix. In that case each
row in $A$ would correspond to a single equation with $n$ unknowns. Such an equation defines a
$n-1$ dimensional hyperplane in the $n$ dimensional space with coordinates $x_1, x_2, \ldots,
x_n$. The intersection of two $n-1$ dimensional hyperplanes defines a $n-2$ dimensional
hyperplane. The intersection of that $n-2$ dimensional hyperplane with another $n-1$ dimensional
hyperplane results in a $n-3$ dimensional hyperplane. As each equation is used, the dimension of
the region of intersection decreases by $1$. When all $n$ equations are considered, the region
of intersection is a zeroth dimensional region---a point. It is the point $(x_1, x_2, \ldots,
x_n)$ that simultaneously satisfies all the equations. Even if $n$ is very large, the system of
equations can still be represented by the simple form $Ax = b$.

Solving $Ax = b$ is, in theory very simple. Just ``divide'' both sides by $A$. More precisely
\begin{eqnarray*}
  Ax         & = & b       \\
  A^{-1}(Ax) & = & A^{-1}b \\
  (A^{-1}A)x & = & A^{-1}b \\
  Ix         & = & A^{-1}b \\
  x          & = & A^{-1}b
\end{eqnarray*}
In particular, we can pre-multiply both sides of the equation by $A^{-1}$ and then use the
associativity of matrix multiplication and the special properties of the identity matrix to
reduce the left hand side of the equation to just $x$. The right side of the equation then
becomes $A^{-1}b$. Two matrices are equal only if their corresponding elements are all equal.
Thus the first element of the $x$ column vector, $x_1$, must be equal to the first element in
the $A^{-1}b$ column vector. This is the solution for the unknown $x_1$. The other unknowns are
given by the other values in the $A^{-1}b$ column vector.

Thus we have this important result: Solving the system of equations is simply a matter of
finding the inverse of the matrix of coefficients, if it exists, and then pre-multiplying the
driving vector by that inverse. Notice that if the driving vector is changed a new solution can
be found with a simple matrix multiplication. The matrix of coefficients does not need to be
re-inverted unless one of $A$'s elements changes. This can be significant in some applications.
As you will see, the driving vector is usually an expression of signals applied to a system
while the matrix of coefficients is usually related to the structure of a system. Computing what
a system does with different inputs is usually a matter of solving the equations with a new
driving vector. Once the coefficients are inverted this is a simple matter.

You know from Section \ref{sec:matrix-inverses} that not all matrices have inverses. What is the
significance of the case when the matrix of coefficients is singular? In that situation the
system of equations has no solution. However, the physical interpretation of this case is quite
interesting and worth looking at more closely. Consider the following system
\begin{displaymath}
\left[
\begin{array}{cc}
 1 & 2 \\
 1 & 2
\end{array}
\right]
\left[
\begin{array}{cc}
x_1 \\
x_2
\end{array}
\right] =
\left[
\begin{array}{cc}
3 \\
4
\end{array}
\right]
\end{displaymath}

The two lines defined by these equations are parallel; they have no point of
intersection\footnote{This might be easier to see if you rearrange each equation into the form
  $x_2 = m x_1 + b$}. Thus there is no point $(x_1, x_2)$ that can satisfy both equations at the
same time. It also happens that the matrix of coefficients is singular. Now consider the system
\begin{displaymath}
\left[
\begin{array}{cc}
 1 & 2 \\
 2 & 4
\end{array}
\right]
\left[
\begin{array}{cc}
x_1 \\
x_2
\end{array}
\right] =
\left[
\begin{array}{cc}
1 \\
2
\end{array}
\right]
\end{displaymath}

Here one equation is simply a scaled version of the other. As a result they are really the same
equation. This system does not contain enough information to have a unique solution. In effect
it has only one equation. Once again the matrix of coefficients is singular.

Both of these examples have one thing in common. One row of the matrix of coefficients is a
scaled version of another row. In the first example the scaling factor was $1$. In the second
example the scaling factor was $2$. In the first example the values in the driving vector did
not use the same scaling factor and the result was parallel lines. In the second example the
values in the driving vector did use the same scaling factor and the result was two identical
lines (which are also parallel, of course).

% The paragraphs that follow contain quite a few new concepts. Those concepts should probably be
% developed in sections of their own and not just thrown at the reader the way they are here. It
% would also be a good idea to draw a few figures to help clarify these concepts.

This observation can be generalized but it is necessary to first define a new concept. Suppose
$u$ and $v$ are $n$ element row vectors. A \newterm{linear combination} of $u$ and $v$ is any
vector, $w$ such that
\begin{displaymath}
w = a_1 u + a_2 v
\end{displaymath}
where $a_1$ and $a_2$ are any arbitrary scalars. In other words, I can express $w$ in terms of
$u$ and $v$ by scaling each vector and adding the scaled results. Note that $a_1$ and $a_2$
could be negative or zero. If $w$ is a linear combination of $u$ and $v$ then we say that $w$ is
\newterm{linearly dependent} on $u$ and $v$. If you have a vector that can not be expressed as a
linear combination of other vectors we say that it is \newterm{linearly independent} of those
other vectors.

Consider two vectors in a plane that don't point along the same line. There is no way I can
scale one to get the other. Thus the two vectors are linearly independent. However, any other
vector in that plane can be expressed as a linear combination of the first two vectors. Thus any
other vector in that plane is linearly dependent of the first two. Consider the set of all
vectors that can be written as a linear combination of the first two. That set of vectors
constitutes a \newterm{vector space} and the first two, linearly independent vectors, form the
\newterm{basis} of that space. They are said to \newterm{span} the space.

For any particular vector space there are many different sets of basis vectors possible. Any two
linearly independent vectors in a plane can be used as a basis for that plane. It is common in
engineering and physics to use two orthagonal\footnote{perpendicular} vectors of length $1$ as a
basis for the plane. However, it is perfectly possible, in general, to use two oblique vectors
of arbitrary lengths as a basis as well.

Now suppose that you introduce a third vector that points out of the plane spanned by the first
two. There is no way to write this third vector as a linear combination of the others and so it
is linearly independent of them. In fact, we can talk about the space spanned by the three
vectors---the first two and this new one. That space has three dimensions. Hopefully you can see
that the number of basis vectors required to span a space is equal to that space's
dimensionality. In fact, this is the where the very concept of dimensionality comes from.

Let's bring this back to the subject of simultaneous equations. Each equation in an $n \times n$
system defines a $n-1$ dimension hyperplane. In particular, if you regard each equation as a row
vector, the hyperplane that it describes is perpendicular to that vector and passes through the
tip of that vector. If any of the hyperplanes are parallel there can be no solution to the
system. Furthermore if the intersection of any combination of those hyperplanes is parallel to
any of the remaining hyperplanes there can be no solution. This will occur if any row of the
matrix of coefficients can be written as a linear combination of the other rows. In that case,
the system contains insufficient information for a unique solution. The matrix of coefficients
will be singular.

The following is important.

\begin{theorem}
  Let $A$ be an $n \times n$ matrix. $A$ has an inverse if and only if $\det(A) \neq 0$.
  Furthermore $A$ has an inverse if and only if the rows of $A$ are linearly independent.
\end{theorem}

% It would probably be nice to provide a proof of the above.

Since $A$ needs to have an inverse in order to solve the matrix equation $Ax = b$, the theorem
above is directly relevant to solving such systems. However, in general computing the $\det{A}$
is expensive so a somewhat different approach that is more practical is usually taken.

In a realistic computation the coefficients in the matrix of coefficients are not normally known
to infinite precision. If they represent physical quantities they may only be accurate to two or
three significant figures. Even if they coefficients are known to infinite precision, the
computer used to calculate a solution has finite precision. Thus you would not expect the
solution to be perfectly accurate.

If the matrix of coefficients is close to being singular that means that the hyperplane
described by one of its rows is nearly parallel to the intersection of the other hyperplanes. As
a result, very slight variations in the coefficients will cause a very large change in the
solution. Such a system is said to be \newterm{ill-conditioned}. In real calculations, exactly
singular matrices don't exist. Singularity is a matter of degree and, due to practical limits of
precision, one normally only sees systems with relatively more or less ill-conditioning.

Since many naive formulations of real problems lead naturally to ill-conditioned systems, it is
important to be aware of this issue and to check the conditioning of one's system before
expending too many computational resources trying to solve it. Thus we can define a
\newterm{condition number} of a matrix, $\kappa_\infty$, as a figure of merit that we can use to
judge a system's degree of ill-conditioning.
\begin{displaymath}
  \kappa_\infty(A) = \norm{A}_\infty \cdot \norm{A^{-1}}_\infty
\end{displaymath}

In a perfectly conditioned case, all the hyperplanes described by the rows of the matrix are
perpendicular. This occurs for the identity matrix and thus $\kappa_\infty(I_n) = 1$ is the
ideal. For matrices where some of the hyperplanes are oblique $\kappa_\infty$ is greater than
one and approaches infinity for the singular case.

Note that you can compute the condition number of a matrix using a norm other than the
$\infty$-norm. However, the general properties of the condition number remain the same. One
reason why $\kappa_\infty$ is nice to use is because the $\infty$-norm of a matrix is easy to
calculate (see Section~\ref{sec:matrix-norms}). Unfortunately a proper computation of
$\kappa_\infty$ requires that the matrix be inverted. However, if the system is ill-conditioned,
the inverse of $A$ can't be computed accurately. Indeed, the whole point of computing
$\kappa_\infty$ is to know if $A^{-1}$, or an equivalent result, can be calculated with
reasonable accuracy. Much has been written on how to estimate a system's condition number
without computing the inverse of the matrix of coefficients and without spending too much time
doing it. A full discussion of this matter is outside the scope of this short document.

\subsection*{Exercises}

\begin{enumerate}

\item \emph{I need something here.}

\end{enumerate}

\subsection*{Octave}


\section{Gaussian Elimination}
\label{sec:gaussian}

% I might want to define "numerical stability" somewhere before this paragraph. It is a big
% topic and it might even deserve its own section.

Although this document is not intended to cover issues related to numerical computation, there
is one numerical method that is so important that it does deserve coverage: Gaussian
Elimination. This method can be used to solve a system of linear equations in a reasonably
effective manner. Although it requires a running time of $O(n^3)$ it is considerably more
efficient than some other techniques (such as Kramer's Rule using determinants). In addition
Gaussian Elimination has good numerical properties and works well on dense systems\footnote{A
  \newterm{dense} system of equations is one in which a large percentage of the coefficients are
  not zero.}.

In principle Gaussian Elimination is very simple to understand. Three basic row operations are
defined.
\begin{enumerate}
\item Exchange two rows, $R_i \leftrightarrow R_j$
\item Multiple a row by a scalar, $R_i \leftarrow a R_i$
\item Replace a row with the sum of that row and another scaled row, $R_j \leftarrow R_j + a R_i$
\end{enumerate}

In each case it is necessary to also apply the operations to the corresponding elements of the
driving vector. It is easy to see that these operations are justified based on concepts from
elementary algebra. For example exchanging two rows is equivalent to writing down the equations
in a different order; this does not change the solution of the system. Furthermore, multiplying a
row by a scalar is equivalent to multiplying an equation by a scalar, and so forth.

The easiest way to explain how Gaussian Elimination works is to show an example. Consider the
following 3x3 system of equations.
\begin{displaymath}
\begin{array}{rcrcrcr}
 1.23\, x_1 & - & 4.53\, x_2 & + & 2.83\, x_3 & = &  6.77  \\
 8.33\, x_1 & + & 1.93\, x_2 & + & 3.28\, x_3 & = & -2.33 \\
-3.48\, x_1 & + & 7.12\, x_2 & - & 1.20\, x_3 & = &  6.12
\end{array}
\end{displaymath}
We can write this system in matrix form as follows
\begin{displaymath}
\left[
\begin{array}{rrr}
  1.23 & -4.53 &  2.83 \\
  8.33 &  1.93 &  3.28 \\
 -3.48 &  7.12 & -1.20 
\end{array}
\right]
\left[
\begin{array}{ccc}
x_1 \\
x_2 \\
x_3
\end{array}
\right] =
\left[
\begin{array}{rrr}
  6.77 \\
 -2.33 \\
  6.12
\end{array}
\right]
\end{displaymath}

The first phase of the computation is the \newterm{elimination} phase. It entails using the
three row operations listed above to change the matrix of coefficients into a upper triangular
matrix. In other words, a matrix in which all the elements below the main diagonal are zero. The
computation proceeds by considering each diagonal element one at a time. Starting with $a_{11}$
we start be scanning down the column below and including the diagonal element looking for the
element with the largest absolute value. In this case row two holds that element ($a_{21} =
8.33$). We thus do $R_1 \leftrightarrow R_2$ to exchange those rows.
\begin{displaymath}
\left[
\begin{array}{rrr}
  8.33 &  1.93 &  3.28 \\
  1.23 & -4.53 &  2.83 \\
 -3.48 &  7.12 & -1.20 
\end{array}
\right]
\left[
\begin{array}{ccc}
x_1 \\
x_2 \\
x_3
\end{array}
\right] =
\left[
\begin{array}{rrr}
 -2.33 \\
  6.77 \\
  6.12
\end{array}
\right]
\end{displaymath}

% Do I want to include a proof of the statement below about accuracy?

This step is not strictly necessary but it turns out to improve the accuracy of the computation.
Notice that the elements in the driving vector are exchanged as well. This is necessary, of
course, because those elements are part of the equations being exchanged.

Next using $a_{11}$ we do row operations to zero out the coefficients below it in the same
column. Specifically we do $-(1.23/8.33)R_1 + R_2 \rightarrow R_2$ and $(3.48/8.33)R_1 + R_3
\rightarrow R_3$. The result is
\begin{displaymath}
\left[
\begin{array}{rrr}
 8.33 &  1.93 & 3.28 \\
 0.00 & -4.82 & 2.35 \\
 0.00 &  7.93 & 0.17 
\end{array}
\right]
\left[
\begin{array}{ccc}
x_1 \\
x_2 \\
x_3
\end{array}
\right] =
\left[
\begin{array}{rrr}
 -2.33 \\
  7.11 \\
  5.15
\end{array}
\right]
\end{displaymath}

Moving on to $a_{22}$ we again search the column beneath and including that element looking for
the element with the largest absolute value. In this case that element is $a_{32} = 7.93$. We
thus do $R_2 \leftrightarrow R_3$ to exchange those rows.
\begin{displaymath}
\left[
\begin{array}{rrr}
 8.33 &  1.93 & 3.28 \\
 0.00 &  7.93 & 0.17 \\
 0.00 & -4.82 & 2.35
\end{array}
\right]
\left[
\begin{array}{ccc}
x_1 \\
x_2 \\
x_3
\end{array}
\right] =
\left[
\begin{array}{rrr}
 -2.33 \\
  5.15 \\
  7.11
\end{array}
\right]
\end{displaymath}

Finally we zero out the coefficients below the (new) $a_{22}$ by doing $(4.82/7.93)R_2 + R_3
\rightarrow R_3$.
\begin{displaymath}
\left[
\begin{array}{rrr}
 8.33 & 1.93 & 3.28 \\
 0.00 & 7.93 & 0.17 \\
 0.00 & 0.00 & 2.45
\end{array}
\right]
\left[
\begin{array}{ccc}
x_1 \\
x_2 \\
x_3
\end{array}
\right] =
\left[
\begin{array}{rrr}
 -2.33 \\
  5.15 \\
 10.24
\end{array}
\right]
\end{displaymath}

This completes the elimination phase. The next phase is called \newterm{back substitution}. We
start by observing that the last row represents the equation $2.45\,x_3 = 10.24$. From this
equation we can immediately calculate $x_3 = 10.24/2.45 = 4.18$. Notice also that the second to
last row can be written as $7.93\,x_2 + 0.17\,x_3 = 5.15$. Of course this equation can be
written as $x_2 = (5.15 - 0.17\,x_3)/7.93$. Given that $x_3$ has just been calculated we can now
compute $x_2 = 0.56$. Similarly $x_1 = (-2.33 - 1.93\,x_2 - 3.28\,x_3)/8.33 = -2.06$.

The back substitution phase can store the computed results in the space used for the driving
vector. After each value is computed the corresponding driving vector slot is no longer needed
and can be used for storing the final result.

If you substitute the values for $(x_1, x_2, x_3)$ computed above back into the original system
you will find that they \emph{almost} work. The error you observe is due to the imprecise nature
of the computations done. In this example only two or three significant figures are retained in
each step. As with any numerical computation, errors can potentially accumulate as the
computation proceeds. In misbehaving situations, the resulting accumulated errors can be so
great as to swamp out any real results. Because it is such an important algorithm, Gaussian
Elimination has been extensively studied with respect to its error propagation behavior.
However, it is outside the scope of this document to discuss that issue any further.

It is important to note that the elimination phase of the computation runs in $O(n^3)$ time,
where $n$ is the number of equations being solved. This is because each diagonal element must be
considered and, for each such element an average of $n/2$ rows must be processed and, for each
row $O(n)$ computations are needed. In contrast, the back substitution phase only requires
$O(n^2)$ time. This is because $n$ rows must be processed and, for each row an average of
$O(n/2)$ computations are needed. This means for large systems, the dominate factor limiting the
performance of the algorithm is the elimination phase.

\subsection*{Exercises}

\begin{enumerate}

\item \emph{I need something here.}

\end{enumerate}

\subsection*{Octave}


\section{Eigenvalues and Eigenvectors}

\subsection*{Exercises}

\begin{enumerate}

\item \emph{I need something here.}

\end{enumerate}

\subsection*{Octave}


\end{document}
