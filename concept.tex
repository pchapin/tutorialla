
\section{The Concept of a Matrix}
\label{sec:matrix-concept}

A \newterm{matrix} is, essentially, a rectangular array of numbers. It is the basic ``data
type'' used in linear algebra. The individual numbers in a matrix, usually called the
\newterm{elements}, can be drawn from any of the usual sets. In many applications the matrix
elements are real numbers but in some cases they might be complex numbers, integers, or rational
numbers. We speak of real matrices, complex matrices, and so forth according to the type of the
matrix elements.

The dimensions of a matrix are usually given in the form $m \times n$ where $m$ is the number of
rows and $n$ is the number of columns. The row dimension is always given first. If $n = m$ then
the matrix is said, for obvious reasons, to be square. Here is an example of how a $2 \times 3$
matrix named $A$ might be displayed.

\begin{displaymath}
A = \left[
\begin{array}{ccc}
a_{11} & a_{12} & a_{13} \\
a_{21} & a_{22} & a_{23}
\end{array}
    \right]
\end{displaymath}

Each element is given two subscripts to indicate its position in the matrix. The first subscript
is the row number, ranging from $1\ldots m$, and the second subscript is the column number,
ranging from $1\ldots n$. Typically when one wants to refer to a generic element of matrix one
would use variables such as $i$, $j$, or $k$ to be placeholders for the indices. For example, a
typical element of matrix $A$ might be written as $a_{ij}$. As a computer programmer you should
be able to relate this notation to the index variables used in typical loops. For example, in C
\begin{verbatim}
double A[2][3];

for (int i = 0; i < 2; i++) {
  for (int j = 0; j < 3; j++) {
    A[i][j] = ... ;
  }
}
\end{verbatim}

Notice that in the mathematical notation it is traditional to regard the matrix subscripts as
starting at one and not zero. Also in the mathematical notation it is traditional to use
uppercase letters to represent the entire matrix and lowercase letters to represent individual
matrix elements.

Matrices with only one row, that is $1 \times n$ matrices, are often called \newterm{row
  vectors}. In many cases the row subscript is dropped, since it is always $1$, when talking
about the elements of a row vector. Similarly matrices with only one column are often called
\newterm{column vectors}. In that case the column subscript is often dropped when talking about
the elements of a column vector. Row vectors and column vectors are essentially ordinary arrays
in a programming sense. In the mathematics of linear algebra they are basically treated like any
other matrix although their unusual dimensions give them somewhat special properties.

You might have noticed that in the notation above I did not put any punctuation between the two
subscripts in, for example, $a_{12}$. This might cause you to wonder what happens if one or both
of the subscripts exceeds $9$. In general this is not a significant concern. In the mathematical
development one rarely uses specific numbers other than $1$ or $2$ anyway. For example, I might
show a general $m \times n$ matrix as follows

\begin{displaymath}
A = \left[
\begin{array}{cccc}
  a_{11} & a_{12} & \cdots & a_{1n} \\
  a_{21} & a_{22} & \cdots & a_{2n} \\
  \vdots & \vdots & \ddots & \vdots \\
  a_{m1} & a_{m2} & \cdots & a_{mn}
\end{array}
    \right]
\end{displaymath}

In the discussion about this matrix I might talk about element $a_{ij}$ and how, for example, it
relates to $a_{ii}$ or $a_{mj}$ or even $a_{i+1,j-1}$. However, it would be extremely rare for
me to talk about a specific element such as $a_{3982}$ so the ambiguity about what $3982$ means
in this case would not be an issue.

Keep in mind also that many of the applications of linear algebra involve the use of very large
matrices. While the examples in this document and most text books are small so that they can be
worked by hand, in the real world matrices with dimensions of many thousands of rows and columns
are commonplace. The theory, of course, has no special problems with such large matrices but
doing computations efficiently on them can be a serious concern. Much has been written about
that subject but it is outside the scope of this document.

\subsection*{Exercises}

\begin{enumerate}

\item \emph{I need something here.}

\end{enumerate}

\subsection*{Octave}

I will assume here that you will be using the open source program Octave for your computer
experiments. However, if you have access to MATLAB you should find that all of the commands I
talk about here will also work for you. The two programs have a substantial degree of
compatibility, particularly in the area of basic commands.

When you start Octave you will get a prompt that might look something like

\begin{verbatim}
octave:1>
\end{verbatim}

The number (shown as ``1'' above) is the command number. It increments after each command
although in this document I will typically show it always as ``1.'' You can use octave as a
simple calculator, just type in expressions for it to evaluate. For example

\begin{verbatim}
octave:1> 2 + 10 / 5
ans = 4
octave:2>
\end{verbatim}

Octave stores the result of the evaluation into a special variable named \texttt{ans} that it
displays on the next line. After displaying the answer it then prints a new prompt. You can
store the result of an expression into a named variable using the obvious syntax.

\begin{verbatim}
octave:1> x = 2 + 10 / 5
x = 4
\end{verbatim}

The beauty of Octave is that it allows you to manupulate matrices in a natural way. To enter a
matrix use square brackets.

\begin{verbatim}
octave:1> A = [ 1 2 3; 4 5 6; 7 8 9 ]
A =

  1   2   3
  4   5   6
  7   8   9
\end{verbatim}

Octave allows a fairly flexible input syntax. You can separate matrix elements with white space
alone or with commas at your option. The semicolons are used to delimit one row from another.
However, you could have also entered the matrix on multiple lines using the RETURN key at the
end of each row in a natural way.

Once a matrix has been entered you can access an individual element using parenthesis. For
example \texttt{A(1,3)} prints out the value of $a_{13}$. You can also modify a particular
matrix element.

\begin{verbatim}
octave:2> A(1,3) = 10
A =

  1   2  10
  4   5   6
  7   8   9
\end{verbatim}

Notice that when a matrix element is modified, Octave redisplays the entire matrix.

There are some quirks with Octave to keep in mind as you use it. First, as is mathematically
traditional, element indices are numbered starting at $1$, not $0$ as a C programmer might
expect. Also Octave uses parenthesis instead of square brackets when accessing an individual
element. These things are consistent with Fortran's syntax and are considered natural among
numerical specialists who use Fortran. Also be aware that Octave uses a floating point type
internally to hold all scalars. This is typically what you want but it means that its integer
calculations are somewhat of an illusion. Be careful.

You should take Octave for a test drive. Try entering in a few matrices and accessing their
elements. Although I didn't discuss it here, experiment with Octave's slice notation. For
example, using the matrix $A$ entered above, try executing the command \texttt{A(2:3, 2)}. Can
you explain what happens? Verify your explaination by trying some other possibilities. Also try
\texttt{A(:, 2)} as well.
