
\section{Matrix Inverses}
\label{sec:matrix-inverses}

In order to talk about the inverse of a matrix I should first talk about the concept of
identities. Consider ordinary multiplication. There exists a number, namely $1$, that has the
property $x \times 1 = 1 \times x = x$. Thus $1$ is called the \newterm{multiplicative identity}
because when you multiply any number by $1$ the result is the same. It turns out that there is a
\newterm{identity matrix} as well. Actually there are infinitely many identity matrices of
different sizes. The general form is
\begin{displaymath}
I_n = \left[
\begin{array}{cccc}
 1 & 0 & \cdots & 0 \\
 0 & 1 & \cdots & 0 \\
 \vdots & \vdots & \ddots & \vdots \\
 0 & 0 & \cdots & 1
\end{array}
\right]
\end{displaymath}

That is, $I_n$ is an $n \times n$ square matrix that has ones down the diagonal and zeros
everywhere else. It is relatively easy to see that $IA = A$ for any matrix $A$ (just try
multiple out an example and you will quickly see how it works). Note that $A$ need not be
square. As long as you use the proper sized identity matrix so that the multiplication is
defined the result of $IA$ will be $A$ again.

The identity matrix is also special because, unlike the general case, multiplication by the
identity martix commutes. That is $IA = AI = A$. This assumes that $A$ is square. Otherwise one
must use different sized identities depending on if the identity is being pre-multiplied or
post-multiplied.

Another property of real numbers is that every number except zero has a \newterm{multiplicative
  inverse}. For example $x$'s inverse, symbolized as $x^{-1}$ is such that $xx^{-1} = x^{-1}x =
1$. Similarly many matrices have inverses as well. For example $A$'s inverse, symobolized as
$A^{-1}$ is such that $AA^{-1} = A^{-1}A = I$. Here I restrict myself to square matrices. In
that case notice how multiplication by the inverse commutes; again contrary to the general case
for matrix multiplication.

% It might be useful to prove some of these properties.

With real numbers we can define division in terms of multiplication by an inverse. Specifically
we can define $a/b$ as $ab^{-1}$ or as $b^{-1}a$. Here I'm taking advantage of the fact that
multiplication of real numbers commutes in every case. Similarly in situations where one is
tempted to divide by a matrix we normally talk instead about multiplying by the inverse of the
matrix instead. Because matrix multiplication is not commutative we have the added complication,
in the algebra of matrices, of having to carefully track pre-multiplications as opposed to
post-multiplications. An example of this will be shown in the next section.

Computing the inverse of a matrix is a non-trivial problem. I will discuss this more in a later
version of this document.

% Not all matrices have inverses. Some matrices are singular. Define and discuss this. It might
% also be necessary to talk about the det() operation and how it relates to the existence of an
% inverse.

\subsection*{Exercises}

\begin{enumerate}

\item Consider the following $n \times n$ matrix.
\begin{displaymath}
A = \left[
\begin{array}{cccc}
  a_{11} & 0 & \cdots & 0 \\
  0 & a_{22} & \cdots & 0 \\
  \vdots & \vdots & \ddots & \vdots \\
  0 & 0 & \cdots & a_{nn}
\end{array}
    \right]
\end{displaymath}
This matrix has non-zero elements only on its main diagonal. What is its inverse?

\end{enumerate}

\subsection*{Octave}

You can use the \texttt{eye()} function in Octave to generate an identity matrix of any size.
For example \texttt{eye(3)} returns a $3 \times 3$ identity. You can also use the \texttt{inv()}
function to compute the inverse of a matrix. For example

\begin{verbatim}
octave:1> A = [ 1 2 4; -4 2 1; 9 -2 1 ];
octave:2> B = inv(A)
\end{verbatim}

Verify that the result of \texttt{inv()} is a proper inverse by computing $A*B$ and $B*A$ and
showing that both result in the identity matrix. You may notice that the results shows slight
round off errors in the computations (mostly identified by the existence of negative zeros in
the result). This is normal.

What does Octave do when asked to find the inverse of the singular matrix in the example above?
